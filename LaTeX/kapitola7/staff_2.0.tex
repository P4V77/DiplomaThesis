\newpage
\chapter{Aplikace Staff 2.0}

Aplikace \textit{Staff 2.0} primárně slouží ke konfiguraci a řízení výstupních proudů zařízení \emph{MoSeZ}.
Kromě toho implementuje mechanismy pro detekci a správu připojených zařízení prostřednictvím sériové komunikace.
Aktivně monitoruje příchozí zprávy, na jejichž základě rozpoznává přítomnost a aktivitu jednotlivých jednotek.

Vývoj aplikace vycházel z první verze \textit{Staff}, která byla vytvořena pro prototyp zařízení \emph{MoSeZ}.
Pro tvorbu obou verzí aplikací bylo použito vývojové prostředí \textit{Matlab App Designer}.


\section{Obsluha aplikace}


Po spuštění počítače je nejprve nutné zvolit ve výběrovém seznamu (\textit{droplistu}) správný komunikační port,
který odpovídá připojenému převodníku \textit{USB-CAN} nebo \textit{USB-UART}.
Po výběru portu uživatel klikne na tlačítko \textit{Connect}, čímž aplikace zahájí sledování zpráv na sériovém rozhraní.
Zprávy mohou být přenášeny přímo přes \textit{CAN} nebo se může jednat o výpis zpráv z \textit{CAN} na \textit{UART}.

Aplikace na základě příchozích zpráv automaticky rozpozná připojená zařízení a pro každé z nich vytvoří odpovídající ovládací prvky v uživatelském rozhraní.
Pro každý zdroj může uživatel konfigurovat parametry jako výstupní proud, typ výstupu (\textit{DC/AC}), frekvenci, fázi, offset, a další.

Zapnutí a vypnutí konkrétního zdroje se provádí kliknutím na jeho \textit{ID label} v grafickém rozhraní.
V případě, že některé z připojených zařízení přestane pravidelně odesílat zprávy typu \textit{Heartbeat},
je po uplynutí definovaného časového limitu (aktuálně 15 sekund) automaticky označeno jako neaktivní a jeho ovládací prvky jsou z rozhraní odstraněny.

Tento mechanismus zajišťuje, že uživatelské rozhraní vždy odráží aktuální stav připojených zařízení,
a umožňuje dynamickou správu měnícího se počtu jednotek v síti bez nutnosti ručního zásahu.


\section{Detekce připojených zařízení}

Klíčovou roli v příjmu a zpracování dat ze sériové linky hraje funkce \texttt{serialRx}.
Tato metoda dekóduje přicházející zprávy a hledá identifikátory zařízení (\textit{ID label}), které slouží k automatické detekci připojených jednotek \emph{MoSeZ}.

\subsection{Formáty zpráv}

Pro účely detekce zařízení se využívají následující formáty zpráv:

\begin{table}[h!]
	\centering
	\caption{Přehled formátů zpráv použitých k detekci zařízení}
	\begin{tabular}{|l|l|l|}
		\hline
		\textbf{Typ zprávy} & \textbf{Formát}             & \textbf{Příklad}          \\ \hline
		Heartbeat           & \texttt{254HB<ID>}          & \texttt{254HB3}           \\ \hline
		Zpráva s ID         & \texttt{<ID>ID...}          & \texttt{3IDxx}            \\ \hline
		Teplotní zpráva     & \texttt{<ID>ID Temp: <v> C} & \texttt{3ID Temp: 27.5 C} \\ \hline
	\end{tabular}
	\label{tab:mosz_messages}
\end{table}

\begin{figure}[htbp]
	\centering
	\includegraphics[width=0.9\textwidth]{kapitola7/Figures/staff_connecting_sources.png}
	\caption{Ukázka přidávání zdrojů v aplikaci Staff 2.0 na základě činnosti na \textit{CAN} sběrnici}
	\label{fig:aplikace_staff_connecting}
\end{figure}

Po rozpoznání ID zařízení ve zprávě je volána funkce \texttt{handleDeviceMessage},
která aktualizuje vnitřní stav aplikace. Konkrétně aktualizuje pole \texttt{connected\_sources},
které udržuje informace o aktuálně připojených jednotkách.
Při detekci nového zařízení aplikace dynamicky vytvoří v uživatelském rozhraní odpovídající ovládací prvky pomocí funkce \texttt{addNewMosezRow}.,
například:

\begin{itemize}
	\item Přepínací tlačítko pro aktivaci/deaktivaci výstupu.
	\item Spinner pro nastavení proudu.
	\item Tlačítka pro volbu výstupu DC/AC.
	\item Volbu průběhu pro AC výstup.
	\item Spinnery pro parametry jako frekvence, offset, fáze a u obdélníku navíc střída a sklon hran.
\end{itemize}

\section{Zpracování neaktivity a odstranění zařízení}

Detekce neaktivních zařízení je zajištěna pomocí periodického časovače \texttt{updateTimer},
který spouští metodu \texttt{updateSourceRows}.
Ta kontroluje čas poslední zprávy ze zařízení a porovnává jej s prahovou hodnotou \texttt{inactivityThreshold} (aktuálně 15 sekund).
Pokud je zařízení neaktivní, volá se \texttt{deleteMoSeZRow}, čímž jsou jeho ovládací prvky odstraněny z GUI.

\begin{figure}[htbp]
	\centering
	\includegraphics[width=0.9\textwidth]{kapitola7/Figures/staff_sources_timeout.png}
	\caption{Ukázka odstranění neaktivních zařízení v aplikaci Staff 2.0}
	\label{fig:aplikace_staff_disconnecting}
\end{figure}

Dále funkce \texttt{compactGridLayout} a \texttt{sortSourceRows} dynamicky upravují rozložení GUI při změně počtu zařízení.
Výšku jednotlivých řádků upravuje \texttt{updateGridRowHeights},
čímž je zajištěno správné zobrazení ovládacích prvků i při velkém počtu připojených jednotek.

\section{Možnosti rozšíření}

Aplikace by rovněž mohla zobrazovat historická data (např.  proudové a teplotní křivky) nebo graficky zvýrazňovat stav zařízení.
Vylepšené hlášení chyb a systém notifikací by dále zlepšily použitelnost.
Pokud by bylo potřeba provozovat aplikaci i na počítačích bez \textit{Matlabu},
bylo by vhodné opustit \textit{Matlab App Designer} a přejít na alternativní platformu,
jako například jako například \textit{Python} s knihovnami \textit{PyQt} nebo \textit{Tkinter} pro desktopové \textit{GUI},
nebo \textit{C\#} s frameworky \textit{.NET MAUI} či \textit{WPF} pro multiplatformní desktopové aplikace.
