% --------------------------
% --- Základní nastavení ---
% --------------------------
% Nastavení jazyka
\usepackage[czech]{babel}      % Czech language support
\usepackage[utf8]{inputenc}
\usepackage[T1]{fontenc}       % Proper Czech font encoding

\usepackage{csquotes}
\usepackage{amsmath,amssymb,latexsym,amsfonts}
% \usepackage{todonotes}
\usepackage{placeins} % Přidat do preambule
\usepackage{tabularx}
\usepackage{array}
\usepackage{pdfpages}  % Make sure this is in your preamble



% Přidání vlastních proměnných (jméno autora apod.)
\newcommand{\thesistype}{Diplomová práce}
\newcommand{\thesistopic}{Rozšíření modulární platformy pro elektromagnetické aktuátory}
\newcommand{\thesisdepartment}{KEP}
\newcommand{\thesisauthor}{Bc. Pavel Michalecký}
\newcommand{\thesissupervisor}{Ing. Martin Vítek}
\newcommand{\thesisconsultant}{Jméno konzultanta}





% ---------------------------
% --- Nastavení dokumentu ---
% ---------------------------
\usepackage{pdfpages}	% Obrázek přes celou stránku
\usepackage[top=2.3cm, left=2.5cm, right=2.5cm, bottom=2.3cm]{geometry}	% Definuje vzhled strany

\renewcommand{\baselinestretch}{1.5}
\setlength{\parskip}{1em} % Nastaveni mezer mezi odstavci
% \setlength{\parindent}{0pt} % Odstraní odsazení odstavců

\usepackage[nottoc,numbib]{tocbibind}	% Přidá odkaz na literaturu do obsahu
\usepackage{tocloft}	% Seznam obrázků, tabulek, obsah nemá číslo

% Definice barev
\definecolor{blue-light}{RGB}{91, 155, 213}
\definecolor{blue-FEL}{RGB}{7, 67, 145}
\definecolor{blue-dark}{RGB}{3, 31, 68}

\usepackage{indentfirst}	% Odsazeni prvního odstavce
\setcounter{secnumdepth}{3}	% Nastavení 3 podúrovní číslování nadpisu
\setcounter{tocdepth}{2}	% Nastaveni 2 podúrovní číslování nadpisu do obsahu
\renewcommand\labelitemi{--} % Nastaveni pomlčky

\usepackage{titlesec}		% Změna třídy chapter
\titlespacing*{\chapter}{0pt}{-25pt}{30pt}
\titleformat{\chapter}[hang]{\Huge\bfseries}{\thechapter\hspace{20pt}{}\hspace{20pt}}{0pt}{\Huge\bfseries}

% ---------------------------
% -- Obrázky, tabulky, ... --
% ---------------------------
\usepackage{graphicx}	% Vkládání obrázků, barvy
\usepackage{float}		% Možné umístění obrázků přesně kam chci + minipage
\usepackage[labelfont=bf]{caption} % Popisky obrázků s tučným úvodem
%\usepackage{url}
\usepackage{wrapfig}	% Obtékání textu kolem obrázku

\usepackage{tabularx}	% Lepší formátování tabulek
\usepackage{array}		% Lepší tabulky (zarovnání, šířka)
\usepackage{longtable}
\usepackage{booktabs}     % hezčí čáry


% Číslování obrázků, tabulek a rovnic 1,2,3 namísto 1.1, 1.2, 2.1
\renewcommand{\thefigure}{\arabic{figure}}
\renewcommand{\thetable}{\arabic{table}}
\renewcommand{\theequation}{\arabic{equation}}

% Čísla obrázků a tabulek rostou napříč více kapitolami (default je, že s každou sekcí se začíná od 1)
\usepackage{chngcntr}
\counterwithout{figure}{chapter}
\counterwithout{table}{chapter}
\counterwithout{equation}{chapter}

\usepackage{fancyref}	% Odkazy

\usepackage{listings}	% Kódy
\usepackage[framed,numbered,autolinebreaks,useliterate]{mcode}	% Formátování matlab kódu

% ---------------------------
% --- Jednotky a veličiny ---
% ---------------------------
\usepackage{siunitx}	% Formátování čísel s jednotkou (řeší nezalomitelné mezery, správné formátování jednotek apod.
\sisetup{product-units = single}	% Při dvou hodnotách vypíše jednotku jen na konci
\sisetup{mode=text,range-phrase = {\text{~až~}}, range-units=single}	% Při rozsahu vypíše mezi hodnotami "až"
\sisetup{output-decimal-marker = {,}}

\usepackage{textcomp}	% Aby gynsymb neukazoval error s \perthousand and \micro
\usepackage{gensymb}	% Symboly jako degree, micro atd.

% ---------------------------
% - Záhlaví, zápatí stránek -
% ---------------------------
\usepackage{fancyhdr}
\headheight 24pt	% nastavení výšky záhlaví

\renewcommand{\headrule}{\color{blue-light}{\rule{\textwidth}{0.5pt}}} % Redefinice barvy linky v záhlaví

\fancypagestyle{plain}{%
	\fancyhf{}

	\fancyhead[L]{{\scriptsize {\thesistopic}}}
	\fancyhead[R]{{\scriptsize {\thesisauthor, } \number\year}}

	\fancyfoot[C]{\thepage}
}

% -------------------------
% -- Nastavení řádkování --
% -------------------------
\usepackage{setspace}
%\singlespacing %jednoduche radkovani
\onehalfspacing %jedna a pul radkovani
%\doublespacing %dvojite radkovani

% --------------------------
% -- Nastavení literatury --
% --------------------------
\usepackage[
%	backend=bibtex,
	style=iso-numeric
]{biblatex}
\addbibresource{bib.bib}

% -------------------------------------
% - Nastavení výstupního PDF a odkazů -
% -------------------------------------
\usepackage[unicode]{hyperref} % měl by být načten jako jeden z posledních https://tex.stackexchange.com/questions/1863/which-packages-should-be-loaded-after-hyperref-instead-of-before
\hypersetup{
	colorlinks=true,
    linkcolor=blue-dark,
	citecolor=blue-dark,
	urlcolor=blue-FEL,
	pdfauthor={\thesisauthor},
	pdftitle={\thesistopic}
}
% --------------------------

%#######################################################################
%#   commands.tex  - Vlastní přikazy pro zpřehlednění výsledné práce   #
%#######################################################################

% -------------------------
% ------- Obrázky ---------
% -------------------------
% nový přikaz  \obr{velikost}{nazevobrazku s priponou}{popis obrazku}
% přiklad \obr{1}{obrazek}{Graf popisujici blabala....}


\DeclareSIUnit{\baud}{Bd}         % unit for modulation rate / line digit rate (13-16)

\newcommand{\obrH}[4]{\begin{figure}[H] % Pramatery H urcuje fixovani pozice obrazku na vlozenem miste 
		\centering
		\includegraphics[scale=#1]{#2}
		\caption{#3}
		\label{obr:#4}
	\end{figure}}

\newcommand{\obr}[4]{\begin{figure}
		\centering
		\includegraphics[scale=#1]{#2}
		\caption{#3}
		\label{obr:#4}
	\end{figure}}

% -------------------------
% --------- TODO ----------
% -------------------------
% Při odkomentování následujícího řádku můžete vkládat velmi jednoduché poznámky
\newcommand\todo[1]{\textcolor{red}{TODO: #1\newline}}

% Pro složitější tvorbu úkolů, odkomentujte následující 4 řádky. V kombinaci s tímto doporučujeme dočasně odkomentovat i styl/seznamy.tex a seznam TODO
%\usepackage[colorinlistoftodos,prependcaption,textsize=tiny]{todonotes}
%\usepackage{regexpatch}
%\makeatletter
%\xpatchcmd{\@todo}{\setkeys{todonotes}{#1}}{\setkeys{todonotes}{inline,#1}}{}{}

% Oba systémy úkolů se následně využívají takto:
%\todo{Chybí tu zmínit měření magnetického pole!}

% -------------------------
% -------- Nadpis ---------
% -------------------------
\newcommand{\nadpis}[1]{
	\noindent{\large{\textbf{#1}}}\\
}

% ----------------------------
%  Nastavení jednotek u rovnic 
% ----------------------------
\makeatletter
\providecommand\add@text{}
\newcommand\tagaddtext[1]{%
	\gdef\add@text{#1\gdef\add@text{}}}% 
\renewcommand\tagform@[1]{%
	\maketag@@@{\llap{\add@text\quad}(\ignorespaces#1\unskip\@@italiccorr)}%
}
\makeatother

\newcommand{\rov}[2]{
	\begin{equation}
		#1 \tagaddtext{($#2$)}
	\end{equation}
}   % Vlozeni vytorenych prikazu, ktere obsahuje soubor commands

% LaTeX Math Definitions

% Complex numbers
\newcommand{\cplx}[1]{{\underline{#1}}} % Complex number
\newcommand{\mi}{\mathrm{i}} % Complex unit
\newcommand{\mj}{\mathrm{j}} % Complex unit
\renewcommand{\Re}{\mathrm{Re}} % Real part
\renewcommand{\Im}{\mathrm{Im}} % Imaginary part

\newcommand{\phas}[1]{{\underline{#1}}} % Phasor
\newcommand{\vecphas}[1]{\mbox{\underline{\boldmath$#1$}}} % Phasor of vector
\newcommand{\faz}[1]{{\underline{#1}}} % Phasor
\newcommand{\vecfaz}[1]{\mbox{\underline{\boldmath$#1$}}} % Phasor of vector

% Vectors and matrices
\renewcommand{\vec}[1]{\mbox{\boldmath$#1$}} % Vector
\newcommand{\mat}[1]{\mathrm{\mathbf{{#1}}}} % Matrix, tenzor

% Diferential operators
\newcommand{\dif}{\,\mathrm{d}} % Differential
\newcommand{\grad}{\mathrm{grad}\ } % Gradient
\newcommand{\curl}{\mathrm{curl}\ } % Rotation
\renewcommand{\div}{\mathrm{div}\ } % Divergence

\newcommand{\laplace}{\triangle} % Laplace

% Others
\newcommand{\const}{\mathrm{const.}} % Constant
\newcommand{\konst}{\mathrm{konst.}} % Constant


\newcommand{\me}{\mathrm{e}} % Euler number

% Czech special definitions
\newcommand{\rot}{\mathrm{rot}\ } % Rotation
\newcommand{\tg}{\mathrm{tg}\ } % Tangens
\newcommand{\arctg}{\mathrm{arctg}\ }  % Vložení definic matematiky

%%%% NAVÍC %%%
% Reference příloh
\newcommand{\pref}[1]{příloha na straně \pageref{#1}, obr. \nameref{#1}}
\newcommand{\rref}[1]{rov. \eqref{#1}}
\newcommand{\oref}[1]{obr. \eqref{#1}}

\newcommand{\Pref}[1]{Příloha na straně \pageref{#1}, obr. \nameref{#1}}
\newcommand{\Rref}[1]{Rov. \eqref{#1}}
\newcommand{\Oref}[1]{Obr. \eqref{#1}}


% Listing
\lstset{literate=
    {á}{{\'a}}1 {č}{{\v{c}}}1 {ď}{{\v{d}}}1 {é}{{\'e}}1 {ě}{{\v{e}}}1 {í}{{\'i}}1 {ň}{{\v{n}}}1 
    {ó}{{\'o}}1 {ř}{{\v{r}}}1 {š}{{\v{s}}}1 {ť}{{\v{t}}}1 {ú}{{\'u}}1 {ů}{{\r{u}}}1 {ý}{{\'y}}1 {ž}{{\v{z}}}1
    {Á}{{\'A}}1 {Č}{{\v{C}}}1 {Ď}{{\v{D}}}1 {É}{{\'E}}1 {Ě}{{\v{E}}}1 {Í}{{\'I}}1 {Ň}{{\v{N}}}1 
    {Ó}{{\'O}}1 {Ř}{{\v{R}}}1 {Š}{{\v{S}}}1 {Ť}{{\v{T}}}1 {Ú}{{\'U}}1 {Ů}{{\r{U}}}1 {Ý}{{\'Y}}1 {Ž}{{\v{Z}}}1
}
