\clearpage

\pagenumbering{roman}

%ABSTRAKT

\chapter*{Abstrakt}

Tato diplomová práce se zabývá návrhem a realizací rozšířené verze \emph{modulární platformy MoSeZ} pro řízení elektromagnetických aktuátorů.
Hlavním cílem bylo rozšíření funkcí a zvýšení spolehlivosti původní verze MoSeZ Rev. A.

Vylepšení zahrnují novou revizi jednokanálového proudového zdroje s výstupním proudem až \SI{8}{\ampere},
podporu střídavého výstupu pomocí přímé digitální syntézy \emph{DDS} a implementaci softwarové \textit{PID} regulace ve firmwaru.

Modulární architektura byla zachována – vícekanálové systémy lze sestavit pomocí sběrnice \textit{CAN} a vlastního komunikačního protokolu.
Praktická část zahrnuje návrh \textit{DPS}, vývoj firmwaru v jazyce \textit{C++} a vývoj aplikace \textit{Staff 2.0} pro konfiguraci proudového výstupu zařízení.

Laboratorní měření prokázala přesnost řízení v režimech DC i AC.
Zařízení bylo úspěšně nasazeno k řízení čtyř voicecoilů ve vibračním podavači firmy \emph{DESSEQ}
a závěrem jsou nastíněny možné směry dalšího vývoje.

\vfill\section*{Klíčová slova}


proudový zdroj, elektromagnetické aktuátory, H-můstek, modulární platforma, PID regulace, přímá digitální syntéza, embedded systémy, CAN sběrnice


\vspace{1cm}

\newpage
\chapter*{Abstract}

This thesis deals with the design and implementation of an extended version of the \emph{modular MoSeZ} platform for controlling electromagnetic actuators.
The main objective was to extend the functionality and improve the reliability of the original MoSeZ Rev. A version.

Enhancements include a new revision of the single-channel current source with an output current up to \SI{8}{\ampere}, 
support for alternating output using direct digital synthesis \textit{DDS}, 
and the implementation of software-based \textit{PID} control in \textit{firmware}.

The modular architecture has been retained — multi-channel systems can be assembled using the \textit{CAN} bus and a custom communication protocol. 
The practical part includes the design of the \textit{PCB}, \textit{firmware} development in \textit{C++}, and the \textit{Staff 2.0} application for configuring the current output.

Laboratory measurements confirmed control precision in both DC and AC modes. 
The system was successfully deployed to control four voice coils in a \emph{DESSEQ} vibratory feeder, 
and possible directions for further development are outlined.


\vfill\section*{Keywords}
\noindent
current source, electromagnetic actuators, H-bridge, modular platform, PID control, direct digital synthesis, embedded systems, CAN bus
\vspace{1cm}

%PODEKOVANI

\newpage
\clearpage

\vspace*{3cm}
\section*{Poděkování}\vspace{1.5em}
\noindent
Rád bych poděkoval vedoucímu diplomové práce, \emph{Ing. Martinu Vítkovi}, za cenné odborné rady, vstřícný přístup a metodické vedení, 
které významně přispěly k úspěšnému dokončení této práce. Dále děkuji společnosti \emph{DESSEQ} za poskytnutí \emph{voicecoilů} a 
experimentálních vibračních stolic a za možnost ověřit návrh v praxi.  Poděkování patří také \emph{Ing. Zdeňku Kubíkovi, Ph.D.}, 
který umožnil měření vyzařování v EMC komoře. 
V neposlední řadě děkuji za sílu a vnitřní klid, které mi byly v průběhu práce dopřány z míst, jež přesahují každodennost.


\newpage
\clearpage

\vspace*{3cm}
\section*{Prohlášení}\vspace{1.5em}
\noindent
Během přípravy této práce autor použil \textit{ChatGPT} a \textit{Gemini AI}
k zlepšení čitelnosti a jazykové kvality rukopisu. 
Po použití tohoto nástroje/služby autor pečlivě zkontroloval a
upravil obsah podle potřeby a přebírá plnou odpovědnost za výsledný obsah práce.



% Tato diplomová práce se zaměřuje na návrh a realizaci rozšířené verze modulární platformy \emph{MoSeZ} určené pro řízení elektromagnetických aktuátorů. 
% Cílem práce bylo zvýšit modularitu, spolehlivost a funkční rozšiřitelnost původní verze zařízení \emph{MoSeZ Rev. A}, 
% a přizpůsobit ji pro širší spektrum technických aplikací.

% Navržená vylepšení zahrnují novou hardwarovou revizi jednokanálového proudového zdroje s výstupním proudem až \SI{8}{\ampere}, 
% doplněnou o podporu střídavých proudových výstupů pomocí přímé digitální syntézy (\emph{DDS}) a 
% řešení regulace pomocí softwarového \emph{PID} regulátoru.

% Významným aspektem návrhu je zachování modulární architektury platformy, 
% která umožňuje sestavení vícekanálového systému z jednotlivých jednokanálových modulů propojených pomocí průmyslové sběrnice \emph{CAN}. 
% Vlastní komunikační protokol založený na sběrnici \emph{CAN} zajišťuje kompatibilitu a integraci do rozsáhlejších systémů.

% Teoretická část práce se zabývá rešerší komunikačních protokolů využívajících sběrnici \emph{CAN}.
% Praktická část zahrnuje návrh desky plošných spojů, vývoj firmwaru v jazyce \emph{C++} a 
% tvorbu desktopové aplikace \emph{Staff 2.0} pro konfiguraci a diagnostiku proudového výstupu zařízení.

% Navržené řešení bylo ověřeno sérií laboratorních měření zaměřených na výstupní proud, 
% účinnost, harmonické zkreslení a přesnost regulace. 
% Výsledky prokázaly, že zařízení splňuje požadavky na přesné řízení elektromagnetických aktuátorů v režimech \emph{DC} i \emph{AC}.
% Zařízení bylo rovněž úspěšně nasazeno pro řízení čtyř voicecoilů v experimentálním vibračním podavači firmy \emph{DESSEQ}. 
% Závěrem je navrženo možné směřování dalšího vývoje platformy.