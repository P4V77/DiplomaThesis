%#######################################################################
%#   commands.tex  - Vlastní přikazy pro zpřehlednění výsledné práce   #
%#######################################################################

% -------------------------
% ------- Obrázky ---------
% -------------------------
% nový přikaz  \obr{velikost}{nazevobrazku s priponou}{popis obrazku}
% přiklad \obr{1}{obrazek}{Graf popisujici blabala....}


\DeclareSIUnit{\baud}{Bd}         % unit for modulation rate / line digit rate (13-16)

\newcommand{\obrH}[4]{\begin{figure}[H] % Pramatery H urcuje fixovani pozice obrazku na vlozenem miste 
		\centering
		\includegraphics[scale=#1]{#2}
		\caption{#3}
		\label{obr:#4}
	\end{figure}}

\newcommand{\obr}[4]{\begin{figure}
		\centering
		\includegraphics[scale=#1]{#2}
		\caption{#3}
		\label{obr:#4}
	\end{figure}}

% -------------------------
% --------- TODO ----------
% -------------------------
% Při odkomentování následujícího řádku můžete vkládat velmi jednoduché poznámky
\newcommand\todo[1]{\textcolor{red}{TODO: #1\newline}}

% Pro složitější tvorbu úkolů, odkomentujte následující 4 řádky. V kombinaci s tímto doporučujeme dočasně odkomentovat i styl/seznamy.tex a seznam TODO
%\usepackage[colorinlistoftodos,prependcaption,textsize=tiny]{todonotes}
%\usepackage{regexpatch}
%\makeatletter
%\xpatchcmd{\@todo}{\setkeys{todonotes}{#1}}{\setkeys{todonotes}{inline,#1}}{}{}

% Oba systémy úkolů se následně využívají takto:
%\todo{Chybí tu zmínit měření magnetického pole!}

% -------------------------
% -------- Nadpis ---------
% -------------------------
\newcommand{\nadpis}[1]{
	\noindent{\large{\textbf{#1}}}\\
}

% ----------------------------
%  Nastavení jednotek u rovnic 
% ----------------------------
\makeatletter
\providecommand\add@text{}
\newcommand\tagaddtext[1]{%
	\gdef\add@text{#1\gdef\add@text{}}}% 
\renewcommand\tagform@[1]{%
	\maketag@@@{\llap{\add@text\quad}(\ignorespaces#1\unskip\@@italiccorr)}%
}
\makeatother

\newcommand{\rov}[2]{
	\begin{equation}
		#1 \tagaddtext{($#2$)}
	\end{equation}
}