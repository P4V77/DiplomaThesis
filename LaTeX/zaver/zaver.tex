\chapter{Závěr}

Tato diplomová práce se zaměřila na návrh, vývoj a testování rozšířené verze modulární platformy \emph{MoSeZ} určené pro řízení elektromagnetických aktuátorů.
Byly navrženy nové hardwarové moduly jednokanálového proudového zdroje s cílem zvýšit výkon, spolehlivost a flexibilitu systému.
Zvláštní důraz byl kladen na konstrukci výstupních proudových obvodů, implementaci bezpečnostních funkcí,
podporu střídavých proudů prostřednictvím \textit{DDS} a rozšíření možností řízení pomocí \textit{PID} regulace.

Platforma \textit{MoSeZ} nadále zachovává svou modulárnost -- jednotlivé jednokanálové moduly lze propojit pomocí sběrnice \textit{CAN} tak,
že se z pohledu uživatele chovají jako vícekanálový proudový zdroj.
Tato vlastnost umožňuje snadnou integraci platformy do větších řídicích systémů.

Těchto vlastností bylo využito při aplikaci ve vibračním podavači firmy \emph{DESSEQ},
kde čtyři moduly \emph{MoSeZ} řízené přes převodník \textit{USB-CAN} napájejí čtyři \textit{voicecoily}.
Pro zajištění bezpečného pohybu jádra bylo implementováno měření polohy pomocí Hallova senzoru,
které umožňuje přímo řídit polohu jádra a tím zabraňuje nárazu jádra do mechanických dorazů.

Dále byla vyvinuta nová desktopová aplikace \textit{Staff 2.0}, 
která poskytuje přehledné grafické rozhraní pro konfiguraci proudových výstupů, 
a ladění systému v reálném čase.
Aplikace automaticky detekuje zařízení dostupná na sběrnici podle jejich identifikátorů
a podporuje komunikaci přes převodníky \textit{USB-UART} i \textit{USB-CAN}.

Funkčnost navrženého systému byla ověřena sérií měření zaměřených na výstupní proudy, harmonické zkreslení, účinnost a přesnost regulace.
Výsledky potvrdily, že zařízení splňuje požadavky na přesné řízení proudových výstupů pro elektromagnetické aktuátory.

Do budoucna lze zvážit implementaci pokročilejších regulačních algoritmů
či návrh nové generace \textit{DPS} optimalizované s ohledem na elektromagnetickou kompatibilitu a platné \textit{EMC} normy.